% Options for packages loaded elsewhere
\PassOptionsToPackage{unicode}{hyperref}
\PassOptionsToPackage{hyphens}{url}
\PassOptionsToPackage{dvipsnames,svgnames,x11names}{xcolor}
%
\documentclass[
]{article}
\title{BDA - Assignment 2}
\author{Anonymous}
\date{}

\usepackage{amsmath,amssymb}
\usepackage{lmodern}
\usepackage{iftex}
\ifPDFTeX
  \usepackage[T1]{fontenc}
  \usepackage[utf8]{inputenc}
  \usepackage{textcomp} % provide euro and other symbols
\else % if luatex or xetex
  \usepackage{unicode-math}
  \defaultfontfeatures{Scale=MatchLowercase}
  \defaultfontfeatures[\rmfamily]{Ligatures=TeX,Scale=1}
\fi
% Use upquote if available, for straight quotes in verbatim environments
\IfFileExists{upquote.sty}{\usepackage{upquote}}{}
\IfFileExists{microtype.sty}{% use microtype if available
  \usepackage[]{microtype}
  \UseMicrotypeSet[protrusion]{basicmath} % disable protrusion for tt fonts
}{}
\makeatletter
\@ifundefined{KOMAClassName}{% if non-KOMA class
  \IfFileExists{parskip.sty}{%
    \usepackage{parskip}
  }{% else
    \setlength{\parindent}{0pt}
    \setlength{\parskip}{6pt plus 2pt minus 1pt}}
}{% if KOMA class
  \KOMAoptions{parskip=half}}
\makeatother
\usepackage{xcolor}
\IfFileExists{xurl.sty}{\usepackage{xurl}}{} % add URL line breaks if available
\IfFileExists{bookmark.sty}{\usepackage{bookmark}}{\usepackage{hyperref}}
\hypersetup{
  pdftitle={BDA - Assignment 2},
  pdfauthor={Anonymous},
  colorlinks=true,
  linkcolor={Maroon},
  filecolor={Maroon},
  citecolor={Blue},
  urlcolor={blue},
  pdfcreator={LaTeX via pandoc}}
\urlstyle{same} % disable monospaced font for URLs
\usepackage[margin=1in]{geometry}
\usepackage{color}
\usepackage{fancyvrb}
\newcommand{\VerbBar}{|}
\newcommand{\VERB}{\Verb[commandchars=\\\{\}]}
\DefineVerbatimEnvironment{Highlighting}{Verbatim}{commandchars=\\\{\}}
% Add ',fontsize=\small' for more characters per line
\usepackage{framed}
\definecolor{shadecolor}{RGB}{248,248,248}
\newenvironment{Shaded}{\begin{snugshade}}{\end{snugshade}}
\newcommand{\AlertTok}[1]{\textcolor[rgb]{0.94,0.16,0.16}{#1}}
\newcommand{\AnnotationTok}[1]{\textcolor[rgb]{0.56,0.35,0.01}{\textbf{\textit{#1}}}}
\newcommand{\AttributeTok}[1]{\textcolor[rgb]{0.77,0.63,0.00}{#1}}
\newcommand{\BaseNTok}[1]{\textcolor[rgb]{0.00,0.00,0.81}{#1}}
\newcommand{\BuiltInTok}[1]{#1}
\newcommand{\CharTok}[1]{\textcolor[rgb]{0.31,0.60,0.02}{#1}}
\newcommand{\CommentTok}[1]{\textcolor[rgb]{0.56,0.35,0.01}{\textit{#1}}}
\newcommand{\CommentVarTok}[1]{\textcolor[rgb]{0.56,0.35,0.01}{\textbf{\textit{#1}}}}
\newcommand{\ConstantTok}[1]{\textcolor[rgb]{0.00,0.00,0.00}{#1}}
\newcommand{\ControlFlowTok}[1]{\textcolor[rgb]{0.13,0.29,0.53}{\textbf{#1}}}
\newcommand{\DataTypeTok}[1]{\textcolor[rgb]{0.13,0.29,0.53}{#1}}
\newcommand{\DecValTok}[1]{\textcolor[rgb]{0.00,0.00,0.81}{#1}}
\newcommand{\DocumentationTok}[1]{\textcolor[rgb]{0.56,0.35,0.01}{\textbf{\textit{#1}}}}
\newcommand{\ErrorTok}[1]{\textcolor[rgb]{0.64,0.00,0.00}{\textbf{#1}}}
\newcommand{\ExtensionTok}[1]{#1}
\newcommand{\FloatTok}[1]{\textcolor[rgb]{0.00,0.00,0.81}{#1}}
\newcommand{\FunctionTok}[1]{\textcolor[rgb]{0.00,0.00,0.00}{#1}}
\newcommand{\ImportTok}[1]{#1}
\newcommand{\InformationTok}[1]{\textcolor[rgb]{0.56,0.35,0.01}{\textbf{\textit{#1}}}}
\newcommand{\KeywordTok}[1]{\textcolor[rgb]{0.13,0.29,0.53}{\textbf{#1}}}
\newcommand{\NormalTok}[1]{#1}
\newcommand{\OperatorTok}[1]{\textcolor[rgb]{0.81,0.36,0.00}{\textbf{#1}}}
\newcommand{\OtherTok}[1]{\textcolor[rgb]{0.56,0.35,0.01}{#1}}
\newcommand{\PreprocessorTok}[1]{\textcolor[rgb]{0.56,0.35,0.01}{\textit{#1}}}
\newcommand{\RegionMarkerTok}[1]{#1}
\newcommand{\SpecialCharTok}[1]{\textcolor[rgb]{0.00,0.00,0.00}{#1}}
\newcommand{\SpecialStringTok}[1]{\textcolor[rgb]{0.31,0.60,0.02}{#1}}
\newcommand{\StringTok}[1]{\textcolor[rgb]{0.31,0.60,0.02}{#1}}
\newcommand{\VariableTok}[1]{\textcolor[rgb]{0.00,0.00,0.00}{#1}}
\newcommand{\VerbatimStringTok}[1]{\textcolor[rgb]{0.31,0.60,0.02}{#1}}
\newcommand{\WarningTok}[1]{\textcolor[rgb]{0.56,0.35,0.01}{\textbf{\textit{#1}}}}
\usepackage{graphicx}
\makeatletter
\def\maxwidth{\ifdim\Gin@nat@width>\linewidth\linewidth\else\Gin@nat@width\fi}
\def\maxheight{\ifdim\Gin@nat@height>\textheight\textheight\else\Gin@nat@height\fi}
\makeatother
% Scale images if necessary, so that they will not overflow the page
% margins by default, and it is still possible to overwrite the defaults
% using explicit options in \includegraphics[width, height, ...]{}
\setkeys{Gin}{width=\maxwidth,height=\maxheight,keepaspectratio}
% Set default figure placement to htbp
\makeatletter
\def\fps@figure{htbp}
\makeatother
\setlength{\emergencystretch}{3em} % prevent overfull lines
\providecommand{\tightlist}{%
  \setlength{\itemsep}{0pt}\setlength{\parskip}{0pt}}
\setcounter{secnumdepth}{-\maxdimen} % remove section numbering
\ifLuaTeX
  \usepackage{selnolig}  % disable illegal ligatures
\fi

\begin{document}
\maketitle

{
\hypersetup{linkcolor=}
\setcounter{tocdepth}{1}
\tableofcontents
}
\begin{Shaded}
\begin{Highlighting}[]
\CommentTok{\# To install aaltobda, see the General information in the assignment.}
\NormalTok{remotes}\SpecialCharTok{::}\FunctionTok{install\_github}\NormalTok{(}\StringTok{"avehtari/BDA\_course\_Aalto"}\NormalTok{, }\AttributeTok{subdir =} \StringTok{"rpackage"}\NormalTok{, }\AttributeTok{upgrade =} \StringTok{"never"}\NormalTok{)}
\end{Highlighting}
\end{Shaded}

\begin{verbatim}
## Skipping install of 'aaltobda' from a github remote, the SHA1 (38f34d35) has not changed since last install.
##   Use `force = TRUE` to force installation
\end{verbatim}

\hypertarget{inference-for-binomial-proportion}{%
\section{Inference for binomial
proportion}\label{inference-for-binomial-proportion}}

\begin{Shaded}
\begin{Highlighting}[]
\CommentTok{\# Installing libraries and setting up dataset}
\FunctionTok{library}\NormalTok{(aaltobda)}
\FunctionTok{data}\NormalTok{(}\StringTok{"algae"}\NormalTok{)}
\end{Highlighting}
\end{Shaded}

\begin{Shaded}
\begin{Highlighting}[]
\CommentTok{\# to check if the code is running correctly (i.e., has been programmed correctly)}
\NormalTok{algaeTest }\OtherTok{\textless{}{-}} \FunctionTok{c}\NormalTok{(}\DecValTok{0}\NormalTok{,}\DecValTok{1}\NormalTok{,}\DecValTok{1}\NormalTok{,}\DecValTok{0}\NormalTok{,}\DecValTok{0}\NormalTok{,}\DecValTok{0}\NormalTok{)}
\end{Highlighting}
\end{Shaded}

\hypertarget{a-and-b}{%
\subsubsection{a) and b)}\label{a-and-b}}

\begin{Shaded}
\begin{Highlighting}[]
\NormalTok{beta\_point\_est }\OtherTok{\textless{}{-}} \ControlFlowTok{function}\NormalTok{(prior\_alpha,}
\NormalTok{                           prior\_beta,}
\NormalTok{                           data) \{}
\NormalTok{  n }\OtherTok{\textless{}{-}} \FunctionTok{length}\NormalTok{(data)}
\NormalTok{  y }\OtherTok{\textless{}{-}} \FunctionTok{sum}\NormalTok{(data)}
\NormalTok{  E\_pay }\OtherTok{\textless{}{-}}\NormalTok{ (prior\_alpha }\SpecialCharTok{+}\NormalTok{ y)}\SpecialCharTok{/}\NormalTok{(prior\_alpha }\SpecialCharTok{+}\NormalTok{ prior\_beta }\SpecialCharTok{+}\NormalTok{ n)}
  \FunctionTok{return}\NormalTok{(E\_pay)}
\NormalTok{\}}

\NormalTok{beta\_interval }\OtherTok{\textless{}{-}} \ControlFlowTok{function}\NormalTok{(prior\_alpha,}
\NormalTok{                          prior\_beta,}
\NormalTok{                          data,}
\NormalTok{                          prob) \{}
\NormalTok{  n }\OtherTok{\textless{}{-}} \FunctionTok{length}\NormalTok{(data)}
\NormalTok{  y }\OtherTok{\textless{}{-}} \FunctionTok{sum}\NormalTok{(data)}
  
\NormalTok{  E\_pay }\OtherTok{\textless{}{-}} \FunctionTok{beta\_point\_est}\NormalTok{(}\AttributeTok{prior\_alpha =} \DecValTok{2}\NormalTok{,}
                          \AttributeTok{prior\_beta  =} \DecValTok{10}\NormalTok{,}
                          \AttributeTok{data =}\NormalTok{ data)}

\NormalTok{  Var\_pay }\OtherTok{\textless{}{-}}\NormalTok{ (E\_pay}\SpecialCharTok{*}\NormalTok{(}\DecValTok{1}\SpecialCharTok{{-}}\NormalTok{E\_pay)) }\SpecialCharTok{/}\NormalTok{ (prior\_alpha }\SpecialCharTok{+}\NormalTok{ prior\_beta }\SpecialCharTok{+}\NormalTok{ n }\SpecialCharTok{+} \DecValTok{1}\NormalTok{)}
\NormalTok{  tstat   }\OtherTok{\textless{}{-}} \FunctionTok{qt}\NormalTok{(}\AttributeTok{p =}\NormalTok{ prob, }\AttributeTok{df =}\NormalTok{ (n}\DecValTok{{-}1}\NormalTok{), }\AttributeTok{lower.tail =} \ConstantTok{TRUE}\NormalTok{)}
  
\NormalTok{  lower }\OtherTok{\textless{}{-}}\NormalTok{ E\_pay }\SpecialCharTok{{-}} \FunctionTok{sqrt}\NormalTok{(Var\_pay}\SpecialCharTok{/}\NormalTok{n)}\SpecialCharTok{*}\NormalTok{tstat }
\NormalTok{  upper }\OtherTok{\textless{}{-}}\NormalTok{ E\_pay }\SpecialCharTok{+} \FunctionTok{sqrt}\NormalTok{(Var\_pay}\SpecialCharTok{/}\NormalTok{n)}\SpecialCharTok{*}\NormalTok{tstat}
  
  \FunctionTok{return}\NormalTok{(}\FunctionTok{list}\NormalTok{(lower, upper))}
  \CommentTok{\#Var\_pay}
\NormalTok{\}}
\FunctionTok{cat}\NormalTok{(}\StringTok{"The following is the expected value of pi"}\NormalTok{)}
\end{Highlighting}
\end{Shaded}

\begin{verbatim}
## The following is the expected value of pi
\end{verbatim}

\begin{Shaded}
\begin{Highlighting}[]
\FunctionTok{cat}\NormalTok{(}\StringTok{"}\SpecialCharTok{\textbackslash{}n}\StringTok{"}\NormalTok{)}
\end{Highlighting}
\end{Shaded}

\begin{Shaded}
\begin{Highlighting}[]
\FunctionTok{beta\_point\_est}\NormalTok{(}\AttributeTok{prior\_alpha =} \DecValTok{2}\NormalTok{,}
               \AttributeTok{prior\_beta  =} \DecValTok{10}\NormalTok{,}
               \AttributeTok{data =}\NormalTok{ algae)}
\end{Highlighting}
\end{Shaded}

\begin{verbatim}
## [1] 0.1608392
\end{verbatim}

\begin{Shaded}
\begin{Highlighting}[]
\FunctionTok{cat}\NormalTok{(}\StringTok{"The following is the confidence interval of pi"}\NormalTok{)}
\end{Highlighting}
\end{Shaded}

\begin{verbatim}
## The following is the confidence interval of pi
\end{verbatim}

\begin{Shaded}
\begin{Highlighting}[]
\FunctionTok{cat}\NormalTok{(}\StringTok{"}\SpecialCharTok{\textbackslash{}n}\StringTok{"}\NormalTok{)}
\end{Highlighting}
\end{Shaded}

\begin{Shaded}
\begin{Highlighting}[]
\FunctionTok{beta\_interval}\NormalTok{(}\AttributeTok{prior\_alpha =} \DecValTok{2}\NormalTok{,}
              \AttributeTok{prior\_beta  =} \DecValTok{10}\NormalTok{,}
              \AttributeTok{data =}\NormalTok{ algae,}
              \AttributeTok{prob =}\NormalTok{ .}\DecValTok{9}\NormalTok{)}
\end{Highlighting}
\end{Shaded}

\begin{verbatim}
## [[1]]
## [1] 0.1591561
## 
## [[2]]
## [1] 0.1625222
\end{verbatim}

\begin{Shaded}
\begin{Highlighting}[]
\FunctionTok{cat}\NormalTok{(}\StringTok{"}\SpecialCharTok{\textbackslash{}n}\StringTok{"}\NormalTok{)}
\end{Highlighting}
\end{Shaded}

\begin{Shaded}
\begin{Highlighting}[]
\FunctionTok{cat}\NormalTok{(}\StringTok{"My answers are different when I did the algaeTest, but they made sense to me. I think I am still thinking in a frequentist manner. Please, explain to me the confidence interval in bayesian."}\NormalTok{)}
\end{Highlighting}
\end{Shaded}

\begin{verbatim}
## My answers are different when I did the algaeTest, but they made sense to me. I think I am still thinking in a frequentist manner. Please, explain to me the confidence interval in bayesian.
\end{verbatim}

\hypertarget{c}{%
\subsubsection{c)}\label{c}}

\begin{Shaded}
\begin{Highlighting}[]
\NormalTok{beta\_low }\OtherTok{\textless{}{-}} \ControlFlowTok{function}\NormalTok{(prior\_alpha,}
\NormalTok{                     prior\_beta,}
\NormalTok{                     data,}
\NormalTok{                     pi\_0) \{}
\NormalTok{  density }\OtherTok{\textless{}{-}} \FunctionTok{integrate}\NormalTok{(}\ControlFlowTok{function}\NormalTok{(pay) }\FunctionTok{dbeta}\NormalTok{(pay, prior\_alpha, prior\_beta), }\DecValTok{0}\NormalTok{,pi\_0)[}\DecValTok{1}\NormalTok{]}
  \FunctionTok{return}\NormalTok{(density)}
\NormalTok{\}}
\FunctionTok{beta\_low}\NormalTok{(}\AttributeTok{prior\_alpha =} \DecValTok{2}\NormalTok{,}
         \AttributeTok{prior\_beta  =} \DecValTok{10}\NormalTok{,}
         \AttributeTok{data =}\NormalTok{ algae,}
         \AttributeTok{pi\_0 =} \FloatTok{0.2}\NormalTok{)}
\end{Highlighting}
\end{Shaded}

\begin{verbatim}
## $value
## [1] 0.6778775
\end{verbatim}

\begin{Shaded}
\begin{Highlighting}[]
\FunctionTok{cat}\NormalTok{(}\StringTok{"}\SpecialCharTok{\textbackslash{}n}\StringTok{"}\NormalTok{)}
\end{Highlighting}
\end{Shaded}

\begin{Shaded}
\begin{Highlighting}[]
\FunctionTok{cat}\NormalTok{(}\StringTok{"The following the is probability of (total mass left of) the historical value of pi = 0.2"}\NormalTok{)}
\end{Highlighting}
\end{Shaded}

\begin{verbatim}
## The following the is probability of (total mass left of) the historical value of pi = 0.2
\end{verbatim}

\end{document}
